The existence of a home advantage has been firmly recognized within the realm of professional sports, with football serving as a prominent example. Prior research has examined diverse factors contributing to this phenomenon, encompassing the impact of crowd pressure on referee decisions, the psychological advantages derived from competing on familiar pitches, as well as the fatigue experienced by away players due to the distance they must travel. However, all of the aforementioned research has focused on home advantage on a team basis. As such, one aspect of home advantage that has not been researched extensively yet is the existence of home advantage on an individual player basis. This paper has investigated the presence and impact of home advantage on an individual player basis. Moreover, this paper has examined how home advantage has evolved over the years. Finally, this study has analyzed the drivers behind home advantage at an individual player level. \\

\noindent
In this paper, 14 years of data relating to the performance of individual players from the English Premier League has been analyzed (2009--2023). More specifically, the data focuses on the home rating advantage (HRA) for a given player in a given season. Here home rating advantage was constructed as the difference between the average rating a player was given for his home games and the average rating a player was given for his away games. Ratings are measured on a continuous scale of 1--10. To investigate the drivers of home advantage at an individual player basis a wide variety of player characteristics and team characteristics were gathered and analyzed. All of the data was webscraped from several websites. \\

\noindent
The analysis revealed that home advantage exists at an individual player basis. When averaging HRA over all players and seasons, a HRA of approximately 0.14 points was found. This result is modest compared to the study by \shortciteA{peeters_2021}, which focused on home advantage at a team level and discovered an average home advantage of 0.64 points per game per team. It is important to note however that this study covered a shorter and more recent timespan. Furthermore, HRA and points gained per game are not directly comparable metrics for measuring home advantage. Still, the notion that \textit{"The whole is greater than the sum of its parts"} seems to hold true in this context. Here the parts refer to the individual players' home advantage and the whole represents the team's home advantage. Next, the findings in this paper indicate a decrease in home advantage over the years. This conclusion is supported by a visual inspection of HRA over the seasons, as well as the inclusion of a trend variable in various models such as linear regression and random forest regression. \\

\noindent
Substantial heterogeneity in the home advantage experienced by individual players was observed. Utilizing a linear regression (LR) framework, several factors that correlate with home advantage were identified. Specifically, average home attendance, promotion to the English Premier League and the skill level of a player exhibit positive correlations. On the other hand, being a goalkeeper or defender and the trend over the seasons exhibit negative correlations with home advantage. \\

\noindent
Machine learning in the form of random forest regression (RFR) was also employed to investigate the drivers of home advantage per player. The results showed that the RFR-based model outperforms the LR-based model in terms of in-sample fit, as demonstrated by various goodness-of-fit metrics such as \textit{$R^2$, MSE} and \textit{MAE}. \\

\noindent
Variable importance analysis revealed that the RFR-based model identifies mostly the same variables as significant in explaining home advantage per player as the LR-based model. The most influential factor is average home attendance, which positively relates to home advantage. Skill moves also exhibit a positive relationship, while the trend is negatively related to home advantage. However, the RFR-based model showed two significant differences compared to the LR-based model. Firstly, the RFR-based model identifies player quality as the second most important explanatory variable in the variable importance analysis. Secondly, the RFR-based model revealed that a player's age is also in the top five most important variables. Both player quality and age were not found to be significant in the LR-based model. Moreover the RFR-based model showed that the relationship between a player's quality and the home advantage he enjoys is highly non-linear. For the lower 50th quantile of player ratings, no discernible effect is observed on home advantage. However, for the upper 50th quantile, a positive exponential effect emerges. For instance, players in the top 20 percent of player ratings enjoy substantially more home advantage than other players. This non-linearity was not captured by the LR-based model. Additionally, age exhibits a negative relationship with the home advantage enjoyed by a player. This result was also not found in the LR-based model. These findings suggest that employing machine learning methods, such as random forest regression, provides additional value in understanding the drivers of home advantage at an individual player level, compared to the more traditional linear regression framework. \\

\noindent
In conclusion, while this study sheds light on player-based home advantage in English professional football, it is important to acknowledge its limitations and identify potential avenues for further research. One notable limitation of this study is the omission of the number of years a player has played for a team in the model. Incorporating this variable could provide valuable insights into the impact of player familiarity with the home environment. Additionally, considering the wage of a player and the transfer fee as factors in the model could offer a deeper understanding of the financial dynamics influencing player performance and the home advantage effect. Furthermore, exploring the robustness of the findings across different countries (e.g., Spain) and competition types (e.g., the UEFA Champions League), would enhance the generalizability of the results and provide a broader perspective on player-based home advantage. Addressing these limitations and pursuing these future research directions will contribute to a more comprehensive understanding of the intricate dynamics underlying home advantage in professional football.