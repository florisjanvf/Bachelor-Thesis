% The roar of the home crowd, the familiarity of the pitch, and the psychological edge of playing on familiar territory - all factors that contribute to the elusive phenomenon of seasonal home advantage in professional football. From the Premier League to the Eredivisie, this age-old phenomenon has fascinated fans and researchers alike. In this paper, we explore the factors that contribute to seasonal home advantage and analyze the extent to which it influences the outcome of matches. Are home teams truly at an advantage or is it just a myth? Join us on this journey to uncover the secrets behind the home team's winning streak.

% When a footballer steps onto the pitch, the pressure to perform is immense. Every pass, every shot, every tackle is scrutinized by fans, coaches, and pundits alike. But what happens when the same player steps onto the pitch in front of their home crowd? Does the familiarity of the stadium and the support of the fans boost their performance? And what happens when they play away from home? In this paper, we dive into the effect and causes of home advantage on individual player performance in matches and seasons. By analyzing data from multiple leagues and competitions, we aim to shed light on the elusive phenomenon of home advantage and its impact on the performance of professional footballers.

When a footballer steps onto the pitch, the pressure to perform is immense. Every pass, every shot, every tackle is scrutinized by fans, coaches, and pundits alike. But what happens when the same player steps onto the pitch in front of their home crowd? Does the familiarity of the stadium and the support of the fans boost their performance? And what happens when they play away from home? In this paper, I dive into the effect and causes of home advantage on individual player performance in matches and seasons. By analyzing data from English professional football, I aim to shed light on the elusive phenomenon of home advantage and its impact on the performance of professional footballers. The main research question in this paper is as follows:
\begin{center}
    %\textit{What is the impact of home advantage on the individual player performance in professional football, and what factors contribute to this effect?}
    \textit{What is the impact of home advantage on individual player performance in professional football?}
\end{center}

\noindent
Football is a constantly evolving sport, and in the past decade, we have seen many changes that have affected the game at every level. From the introduction of Video Assistant Referees (VAR) to the impact of COVID-19, football has undergone significant transformations that have altered the way teams play and the factors that contribute to their success. These changes have also had an impact on home advantage, with empty stadiums becoming the norm during the pandemic and VAR potentially influencing referee decisions. As a result, it is interesting to analyze how the effect of home advantage on individual player performance has developed over the years. Therefore the first subquestion is as follows:
\begin{center}
    \textit{How has the impact of home advantage on individual player performance changed over the years?}
\end{center}

\noindent
Showing the existence of home advantage on an individual player basis is one thing, but explaining what causes it is another. There are many factors at play that may contribute to the effect of home advantage. Is it the familiarity of the stadium, the support of the fans, the position of the player in the field or the age of the player in question? For this subquestion, I dive into what factors contribute to the home advantage effect. Therefore the final subquestion of this paper is as follows:
\begin{center}
    \textit{What factors and player characteristics contribute to home advantage on an individual player basis?}
\end{center}

\noindent
The remainder of this paper is structured as follows. Section \ref{sec:literature review} discusses the relevant literature for the research problem. Section \ref{sec:data} introduces the data used in the quantitative analysis of home advantage. Section \ref{sec:methodology} deals with the methodology of analyzing home advantage on an individual player basis. Section \ref{sec:results} contains the results of the study. Finally, Section \ref{sec:conclusion} summarizes and concludes this paper.