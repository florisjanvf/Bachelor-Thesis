 A substantial amount of research has been conducted on home advantage in professional football. The general consensus from this research is that home advantage is both real and statistically significant. \shortciteA{pollard_1986} investigated the historical progression of home advantage in the English Football League since its inception in 1888 and revealed a striking level of consistency in home advantage until the year 1984. The study also revealed that local derbies and the FA Cup exhibited less of a home advantage effect. Additionally, the research indicated that factors such as crowd size and travel fatigue were not significant contributors to home advantage. Finally, the impacts of team tactics, bias among referees and familiarity with local conditions were inconclusive. After \shortciteA{pollard_1986}, several papers have examined home advantage in English professional football. \shortciteA{barnett_1993} researched the impact of playing on a pitch made from artificial grass rather than natural grass, reporting an extra home advantage of 0.28 points and 0.31 goals per match.\footnote{In 1995 the use of artificial pitches was banned in the English Premier League.} \shortciteA{clarke_1995} investigated seasonal home advantage for all English football teams between 1981 and 1990, their study reveals substantial variation among teams and over time. Furthermore, \shortciteA{bray_2003} discovered that, on average, teams won 22\% more games at home than away in their study on all the top four divisions of the English football league over the time period 1981 to 2000. \shortciteA{carmichael_2005} argue that home advantage relates to differences in playing style, with teams playing more aggressively in home games compared to away games. \shortciteA{dawson_2007} concluded that underdogs are prone to face disciplinary sanctions more frequently compared to title-favorites in the English Premier League, and home teams play more aggressively in front of large crowds, but they do not receive more disciplinary sanctions due to a home team bias. Johnston (2008), however, did not find evidence of referee bias affecting the home advantage. \shortciteA{boyko_2010} claimed that home advantage in the English Premier League is affected by spectator attendance and referee decisions about penalties and yellow cards, while \shortciteA{buraimo_2010} concluded that there is a referee bias favoring home teams. \shortciteA{allen_2014} investigated Premier League matches from season 1992/1993 to season 2011/2012 and found no upward or downward trend in average home advantage over time. Furthermore, they found that teams positioned towards the lower ranks of the league table exhibited a more pronounced home advantage. Contrary to \shortciteA{attrill_2008}, shirt color did not appear to affect home advantage.\\

 \noindent
 The phenomenon of home advantage in professional football has also been investigated in various other countries and in competitions and tournaments of international scope. For instance, studies conducted by \shortciteA{buraimo_2010} focused on the Bundesliga in Germany and \shortciteA{armatas_2014} on the Superleague in Greece. \shortciteA{garicano_2005} find that referees in Spain tend to provide more additional playing time to the home team when they are closely losing. Furthermore, \shortciteA{pollard_2017} explore home advantage in the qualification stage leading up to the FIFA World Cup, discovering that the degree of home advantage was most prominent in Africa and South America and least prominent in Europe. \shortciteA{ponzo_2018} focused on same-stadium derbies in Europe to eliminate the effects of travel distance and familiarity with the stadium. They found that home advantage relies on the crowd's support, which biases referee decisions in benefit of the home team. \shortciteA{krumer_2018} investigated the German Bundesliga and discovered that midweek matches had less home advantage than weekend games due to smaller crowds and players' perception of the game's importance. Similarly, \shortciteA{goller_2020} noted that the day of play affected home advantage in top European leagues. Meanwhile, kick-off times influenced home advantage in the group stage games of the UEFA Europa League, according to \shortciteA{krumer_2020}. \shortciteA{van_ours_2019} found a home advantage of 0.33 points and 0.42 goals per match in Dutch professional football, with artificial pitches providing additional advantage. Furthermore, \shortciteA{van_damme_2019} studied various distance measures in European international football and revealed that altitude and crowd sizes affected home advantage. \shortciteA{amez_2020} concluded that the second match of a knock-out clash did not provide a bigger home advantage, this is noteworthy for competitions such as the UEFA Europa League and UEFA Champions League.\\

 \noindent
 Finally, \shortciteA{peeters_2021} studied seasonal home advantage in English professional football from season 1973/1974 to season 2017/2018. In their paper \shortciteA{peeters_2021} differentiate between absolute home advantage, experienced identically by all teams in a league, and relative home advantage, which varies between teams. Their analysis shows that absolute home advantage is significant, ranging from 0.59 to 0.64 points per game or 0.44 to 0.46 in terms of goal difference. Moreover, there are considerable differences in the relative home advantage enjoyed by clubs, which are positively linked to the variation in attendance within the team and the use of an artificial pitch. Despite large absolute attendance differences across divisions, absolute home advantage remains consistent across divisions. Lastly, \shortciteA{peeters_2021} observe a considerable decrease in absolute home advantage over time that affects all divisions similarly.

\subsection{Scientific Relevance}
The scientific relevance of this paper lies in its focus on the home advantage for individual players on a team, as opposed to the team as a whole. Previous studies have mainly focused on the impact of home advantage on the performance of teams (in the form of points or goals). This study however, looks at the potential benefits that individual players may have in terms of their performance when playing at home. Furthermore, this study seeks to explain which determinants are important in explaining home advantage on an individual player basis. By filling this gap in the literature, the paper provides a more comprehensive understanding of the concept of home advantage in professional football. This research is significant because it adds to the current knowledge on the impact of the home advantage, and may have important implications for how sports teams approach home and away games.

\subsection{Economic Relevance}
The economic and practical relevance of this paper is threefold. Firstly, understanding the factors that contribute to home advantage can help teams and coaches develop strategies to maximize their chances of winning games. For example, if the size of the crowd is found to be a significant factor, teams could try to increase ticket sales or create a more lively atmosphere in their stadium to boost the home advantage. Secondly, home advantage can have financial implications. For instance, teams that perform well at home are more likely to attract fans, which can result in increased ticket sales, merchandise sales, and sponsorship deals. Therefore, understanding the determinants of home advantage can be important for the financial success of a professional football club. Thirdly, home advantage can also have implications for tournament organizers and policymakers. For example, if the type of pitch or the size of the crowd is found to benefit particular players or teams, policymakers could consider implementing changes to ensure fairness and equality in competition.