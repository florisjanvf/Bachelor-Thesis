\begin{table}[htbp]
    \begin{spacing}{1.5}
    \centering
    \small
    \caption{Random forest regression hyperparameter tuning results (full sample)}
    \label{tab:hyperparameter_tuning_results_rfr_full_sample}
    \begin{tabularx}{\textwidth}{p{3cm}p{6cm}p{3cm}p{2cm}}
        \toprule
        \toprule
        \textbf{Hyperparameter} & \textbf{Description} & \textbf{Search space} & \multicolumn{1}{l}{\textbf{Selected value}} \\
        \midrule
        n\_estimators & Number of trees in the forest & [10, 300] & 300 \\
        max\_depth & Maximum depth per tree in the forest & [3, 15] & 7 \\
        max\_features & Maximum number of explanatory variables considered per split in any given tree & sqrt', 'log2', 'all' & \multicolumn{1}{l}{sqrt'} \\
        max\_samples & Number of observations used to train each tree (as a fraction of the total number of observations) & [0.5, 0.7, 0.9, 1.0] & 0.7 \\
        \bottomrule
        \bottomrule
        \multicolumn{4}{X}{\footnotesize\textit{Note.} For the randomized grid search 100 iterations were performed; the random seed for training the random was set at 530; the random seed for the randomized grid search was set at 69.}
    \end{tabularx}
    \end{spacing}
\end{table}

\begin{table}[htbp]
    \begin{spacing}{1.5}
    \centering
    \small
    \caption{Random forest regression hyperparameter tuning results (filtered sample)}
    \label{tab:hyperparameter_tuning_results_rfr_filtered_sample}
    \begin{tabularx}{\textwidth}{p{3cm}p{6cm}p{3cm}p{2cm}}
        \toprule
        \toprule
        \textbf{Hyperparameter} & \textbf{Description} & \textbf{Search space} & \multicolumn{1}{l}{\textbf{Selected value}} \\
        \midrule
        n\_estimators & Number of trees in the forest & [10, 300] & 245 \\
        max\_depth & Maximum depth per tree in the forest & [3, 15] & 7 \\
        max\_features & Maximum number of explanatory variables considered per split in any given tree & sqrt', 'log2', 'all' & \multicolumn{1}{l}{sqrt'} \\
        max\_samples & Number of observations used to train each tree (as a fraction of the total number of observations) & [0.5, 0.7, 0.9, 1.0] & 0.5 \\
        \bottomrule
        \bottomrule
        \multicolumn{4}{X}{\footnotesize\textit{Note.} For the randomized grid search 100 iterations were performed; the random seed for training the random was set at 530; the random seed for the randomized grid search was set at 69.}
    \end{tabularx}
    \end{spacing}
\end{table}