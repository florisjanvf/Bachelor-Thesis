\noindent
This paper studies seasonal home advantage on an individual player basis in the English Premier League over the period 2009 to 2023. Since there are no standardized metrics to objectively measure individual player performance in football, a proxy is constructed in the form of home rating advantage. Home rating advantage per player is constructed using player ratings per game, which are measured on a scale of 1 to 10. The results show that the average home rating advantage over all players over all seasons equals 0.14 points. Moreover, this paper shows that home advantage on an individual player basis has decreased over the years. Furthermore, players differ substantially in the home advantage they enjoy. To investigate this difference further, possible drivers of home advantage per player per season are investigated using both a linear regression framework and a machine learning approach, specifically random forest regression. Home advantage per player is positively related to average home attendance, player quality and skill level and negatively related to age and being a goalkeeper or defender.